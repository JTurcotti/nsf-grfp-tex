\documentclass{nsf-grfp}

\usepackage{lipsum}

\usepackage{hyperref}

\title{Building Friendly Type Systems for Safe Concurrency}
\author{Joshua Turcotti}

\begin{document}

\maketitle

\subsection*{Motivation:} Pervasive use of concurrency is a requirement for modern data-driven platforms such as social networks, messaging services, and e-commerce due to their scale. In building these platforms, correct algorithmic and strategic choices must be made to distribute storage and computation without introducing erroneous behavior like duplicate transactions, security protocol failures, or crashes. All too often, confidence in that correctness comes only from programmers' internal reasoning and, ultimately, from their ability to write tests. This testing-based approach has fundamental flaws because coverage is limited to behaviors that programmers can think to test for, and concurrency bugs in particular can evade even the most thorough attempts at comprehensive testing through their unstable, nondeterministic behavior. The only way to guarantee correct implementations of concurrency is to provide programmers with a language in which the concurrency primitives can be checked for safe usage at compile time. I aim to focus my research on developing type systems, and other static analysis tools, that provide provably safe concurrency without requiring the programmer to provide the logic to do so themselves, without restricting natural programming patterns, and without any runtime effects. I believe that the coming generations of software infrastructure can only be built on such languages, making their development an exciting and fruitful direction for research.

\subsection*{Background:} There have already been many attempts to design languages in this space, with some of the earliest including the Cyclone language developed at Cornell in 2002 \cite{cyclone}. This work demonstrated the use of \textit{linear types} to model sophisticated, flow-sensitive behavior not usually captured by type systems. More recently, linear types have made their way into industry through the growing adoption of the Rust programming language, which enforces safe memory management through linear resources \cite{rust}. Later work such as Haller and Odersky's 2010 system \cite{haller odersky} further specialized the idea of linear logic to the problem of safe concurrency by tracking access capabilities that could be consumed and borrowed to safely perform arbitrary computations on, and subsequently share, data structures. In the early to mid 2010s, many more approaches to the translation of concurrency logic into usable type systems were developed. Pony \cite{pony} is one example that provided a sophisticated hierarchy of capabilities that an object reference could hold, and demonstrated that proper usage of that hierarchy led to safe concurrency through a wide range of natural coding patterns. Work by Marco Servetto including \textit{balloon types} \cite{balloon types} has provided a simpler hierarchy of reference capabilities that restrict the shape of object graphs and possible coding patterns only slightly more. Since most recent languages in this space are not overly restrictive on coding patterns, the largest hindrance to widespread industry adoption of languages with natively-safe concurrency is their annotation burden and intellectual overhead to learn and use. This leaves as an open problem the development of type systems that can still perform robust reasoning about concurrency, but have less impact on the usage experience of the surface-level language.

\subsection*{Research Plan:} My plan to contribute to the design of new languages for safe concurrency is an iterative process of experimentation and human feedback. As part of a joint effort between UC Berkeley's RISELab and Cornell's Applied Programming Languages group, I've recently completed the formalization and proofs of soundness for a novel ownership type system with \textit{regions} \cite{regions}, which I plan to submit to PLDI '22. After building out the tooling and presentation of this system more thoroughly in the beginning of graduate studies, I plan to enter a stage of comprehensive evaluation of the programmer experience in the type system, determining our successes and failures in bringing concurrency reasoning to natural programming patterns. This is an exciting process because I will get the chance to learn so much about the impact of every design decision made throughout the technical development, and by incorporating goals of not just publication, but of human usability and adoption, I believe I can output work with a more streamlined path towards impact and adoption.

In addition to an infusion of human evaluation of my systems within my university, I plan to devote several periods of my education to bringing my work to industry research labs with experience deploying static analysis tools and language updates at scale. Just as a seemingly technically perfect results can be undermined by reception as counterintuitive when new human beings try to use them, tools that are both intuitive and technically powerful can turn out to be impossible to efficiently implement at scale without fundamental revisions. By regularly interfacing my work with industry I will ensure my work is powerful, intuitive, and deployable. Ultimately, I hope to use this cycle to develop a general theory of design principles for safe and usable languages that I will share in a compelling dissertation.

%One significantly under-developed part of this design space is the error messages indicating to a programmer why the type system could not be applied to conclude their code was safe. Existing typecheckers phrase typing errors in terms of the language's own capabilities hierarchy, linear resource system, or restrictive typing rules, which provides little help debugging to any programmer not familiar with that language's internal formalisms. I plan to begin my graduate studies by researching how data races and other failures to correctly implement concurrency can be reported to the programmer without referencing the internal formalisms of the language that detected the failure. If I succeed in tackling and publishing results on the error-reporting problem, I will be in an excellent position to provide other researchers and industry software engineers with a comprehensive set of tooling to experiment with the type system I've developed with Cornell and Berkeley. 

\subsection*{Intellectual Merit:} This work has the potential to advance our understanding of how arbitrary algorithms and data structures can be implemented in the context of a proof obligation for safe concurrency. Since the question of aliasing between arbitrary user-defined variables and object fields, a key component in concurrency reasoning, is undecidable, most ''obvious'' implementations of even simple algorithms cannot be proven safe. In exploring the problems discussed above of designing type systems capable of constructing these proofs, and intelligently reporting failure to construct proofs, new principles concerning algorithm design and implementation will necessarily be discovered. The most direct contribution of this line of work will be providing the academic and industrial communities with safe and friendly languages for concurrent programming. The only path to adoption of such languages is the in-depth development of their type systems to guarantee expressive power, and in-depth development of their surrounding software tooling to provide an efficient programming environment. Each avenue is deeply challenging, but over the next several years, especially with the support of the NSF, I believe that I can make a significant contribution.

\subsection*{Broader Impact:} In my personal statement, I discussed the crucial role of safe and secure programming languages in building a society reliant on its digital infrastructure. Here I'd like to highlight some of the other unique opportunities for impact that will come with my research plan. Programming Languages research is naturally exciting not just to me, but to a whole host of undergraduates inclined to study mathematics, formal logic, and coding at their intersection. Unfortunately, many institutions lack strong undergraduate coursework in the field, and even when efforts are made, the formality of the constructs and the reliance on deep properties of computing and logic can make it intimidating and inaccessible. In particular, it can be difficult to generate research work that can engage and excite undergraduates. My research will involve so much human-facing tooling and investigation that there will be an abundance of exciting opportunities to bring new members of the PL community onboard to contribute to writing example code in fresh languages, seeking better ways to describe concurrent programming patterns, and helping refine the interfaces such as error-reporting. If given NSF funding, I will use it to produce the most robust and frontier-defining technical results I can, but I will also use it to inspire others to produce their very first results, and in the end, it is likely that by doing so I will end up having the far greater impact on the thriving PL community I love.

\begin{thebibliography}{9}

\bibitem{pony}
Sylvan Clebsch, Sophia Drossopoulou, Sebastian Blessing, and Andy McNeil. 2015. \textit{Deny capabilities for safe, fast actors}.
In 5 th Int’l Workshop on Programming Based on Actors, Agents, and Decentralized Control (AGERE!). 1–12. https:
//doi.org/10.1145/2824815.2824816


%\bibitem{vault}
%Manuel Fähndrich and Robert DeLine. 2002. \textit{Adoption and Focus: Practical Linear Types for Imperative Programming}. In
%ACM SIGPLAN Conf. on Programming Language Design and Implementation (PLDI).

\bibitem{haller odersky}
Philipp Haller and Martin Odersky. 2010. \textit{Capabilities for uniqueness and borrowing}. In European Conference on Object-
Oriented Programming. Springer, 354–378.

\bibitem{cyclone}
T. Jim, G. Morrisett, D. Grossman, M. Hicks, J. Cheney, and
Y. Wang. \textit{Cyclone: A safe dialect of C}. In Proceedings of the
USENIX Annual Technical Conference, pages 275–288,
Monterey, California, June 2002.

\bibitem{rust}
Steve Klabnik and Carol Nichols. 2019. \textit{The Rust Programming Language (Covers Rust 2018)}. No Starch Press.

\bibitem{balloon types}
Marco Servetto, David J. Pearce, Lindsay Groves, and Alex Potanin. \textit{Balloon types for safe
parallelisation over arbitrary object graphs}. In WODET 2014 - Workshop on Determinism and
Correctness in Parallel Programming, 2013.

\bibitem{regions}
Mads Tofte and Jean-Pierre Talpin. 1997a. \textit{Region-based memory management}. Information and Computation 132, 2 (1997),
109–176.

\end{thebibliography}

\end{document}
